\chapter{序論}
\label{introduction}

本章では, 本研究の概要を示す. 本研究の動機と, 本論文の構成を述べる. 

\section{はじめに}
\label{introduction:background}
新型コロナウィルスの影響により, 様々な場面でのオンライン化が進んだ. 大学は, 最もオンライン化の影響を受けた環境の一つといえる. 大学教育において, オフラインで行っていた活動をそのままの形式でオンラインでも行うときに, 不便さを感じることがある. 輪読がその代表例といえるだろう. 輪読をオンラインで行うと, 人の発表をただ聞くだけの時間が長く, 集中力が落ちたり聞き逃したりしてしまうことがある. 

また近年, コンストラクティヴィズムの考えに基づいた, 創造的思考力を育む教育方針が見直されている. 

こうした動きと, オンラインという活動環境を得たことで, 従来の輪読という形式よりも文献学習としてより深い理解を得られる方法があるのではないだろうか. 

\section{本論文の構成}

本論文における以降の構成は次の通りである.

~\ref{background}章では,本研究につながる背景について述べる. 
~\ref{issue}章では,本研究における問題定義と, 問題解決における要件について述べる. 
~\ref{proposed}章では,本研究の提案手法を述べる. 
~\ref{implementation}章では,~\ref{proposed}章で述べた提案の正しさを検証するための実験について述べる. 
~\ref{evaluation}章では,\ref{issue}章で求められた課題に対して, 本提案がどの程度解決したかの評価を行い, 考察する. 
~\ref{conclusion}章では,本研究のまとめと今後の課題, 展望について述べる. 

%%% Local Variables:
%%% mode: japanese-latex
%%% TeX-master: "../thesis"
%%% End:
